\chapter{Hardware trojan Insertion/Detection Principle approaches and tools}
\section{Introduction}
\paragraph*{}
In the contemporary landscape of electronic systems and integrated circuits, the increasing complexity and interconnectedness have given rise to unprecedented challenges in ensuring the security and trustworthiness of these critical components. One particularly insidious threat that has garnered significant attention is the insertion of Hardware Trojans (HTs) - clandestine modifications in the hardware design or manufacturing process that compromise the functionality, reliability, or security of the targeted system. As electronic devices play an integral role in our daily lives, ranging from critical infrastructure to personal communication devices, the detection and prevention of Hardware Trojans have become paramount in safeguarding the integrity of these systems.
\paragraph*{}
This chapter explores the sophisticated domain of Hardware Trojans, specifically focusing on the insertion and detection approaches employed in countering this sophisticated threat.The chapter checks not only the techniques employed by adversaries to stealthily introduce Trojans into hardware but also the evolving strategies and technologies designed to identify and mitigate these malicious insertions.Central to this exploration is a comprehensive examination of the software and hardware materials utilized in both the perpetration and defense against Hardware Trojans.
\section{Insertion Approaches}
\paragraph*{}
HT insertion is considered possible at any design phase of an integrated circuit since not all chip third-party suppliers are to be trusted. The first crucial stage in embedded systems design is to gather and analyze the product requirements and turn them into specifications. This phase is the most trusted since suspicious constraints are in the form of text, which eases the process of debugging and checking without any advanced test mechanism.
\paragraph*{}
Whereas, during the design phase, register-transfer level (RTL) is a design abstraction that models a synchronous digital circuit in terms of the flow of digital signals (data) between hardware registers and the logical operations performed on those signals. Register-transfer-level abstraction is used in hardware description languages (HDLs) like Verilog and VHDL to create high-level representations of a circuit, from which lower-level representations and ultimately actual wiring can be derived. For that, untrusted third-party IPs and codes might be used, which would put the entire design at risk.
\paragraph*{}
Logic synthesis is a process by which an abstract specification of desired circuit behavior, typically at register transfer level (RTL), is turned into a design implementation in terms of logic gates, typically by a computer program called a synthesis tool. Common examples of this process include the synthesis of designs specified in hardware description languages, including VHDL and Verilog. Some synthesis tools generate bitstreams for programmable logic devices such as PALs or FPGAs, while others target the creation of ASICs. Besides Place and route is composed of two steps: placement and routing. The first step, placement, involves deciding where to place all electronic components, circuitry, and logic elements in a generally limited amount of space.
\paragraph*{}
This is followed by routing, which determines the exact design of all the wires needed to connect the components. This step must implement all the desired connections while following the rules and limitations of the manufacturing process. At this level of design all related libraries, tools, standard cells, and third-party hard IPs are to be properly certified, so that this step of design would be considered trusted. 
\paragraph*{}
Verification step is intended to check that a product, service, or system meets a set of design specifications. In the development phase, verification procedures involve performing special tests to model or simulate a portion, or the entirety, of a product, service, or system, then performing a review or analysis of the modeling results. In the post-development phase, verification procedures involve regularly repeating tests devised specifically to ensure that the product, service, or system continues to meet the initial design requirements, specifications, and regulations as time progresses. It is a process that is used to evaluate whether a product, service, or system complies with regulations, specifications, or conditions imposed at the start of a development phase. In the contrary with the previous phases, verification step typically performed in the local foundry, which qualify it to be trusted since the use of certified tools and testbenches.
\paragraph*{}
At the fabrication phase, third-party mask shops have access to genuine stream files (GDSII, OASIS), which might be an easy access to system applications and mess with genuine design. The most common attack at the assembly and package phase is to modify authentic hardware components during chip integration and replace them by malicious ones.
\paragraph*{}
In the end, the post-silicon test and validation phase will be safer if it is performed locally in the concerned factory or by another fully certified company instead of outsourcing from an untrusted third-party one, and this phase is considered the last to detect Trojans before delivering the IC for deployment.
% here put figure of design phases where possible ht insertion could be
\paragraph*{}
To assess the risks associated with Hardware trojans (HTs), various studies, such as \cite{insert01} , have presented taxonomies. These taxonomies abstract different categories related to the architecture, effects, and insertion of Trojans in Integrated Circuits (ICs).
\subsection{Insertion phase}
This classification delineates various stages within the IC design process where a potential adversary might be situated. The subsequent analysis outlines the susceptibilities to Trojan insertion at each phase:
\paragraph*{}
\textbf{Specification:} An adversary could intentionally establish weak requirements for the system. This could compromise design reliability, rendering the device susceptible to the leakage of sensitive information.
\paragraph*{}
\textbf{Design:} Even if the entire design is conducted in-house, the use of untrusted tools, libraries, third-party IPs, and standard cells may adversely impact it. For example, untrusted tools might introduce additional circuitry to create backdoors in the authentic design. If any aspect of the design phase is outsourced, a Trojan could be directly incorporated into the hardware description files of the genuine circuit.
\paragraph*{}
\textbf{Fabrication:} A foundry, mask shop, or their staff members who are not trustworthy may gain entry to authentic circuit components, enabling them to anticipate the circuit's functioning and possible uses. This makes the design susceptible to the insertion or deletion of components. Furthermore, modifying the physical characteristics of the circuit, such as sizes and channel doping concentration levels, can significantly increase the susceptibility of the circuit to attacks based on faults.
\paragraph*{}
\textbf{Assembly and package:} The IC is enclosed in a protective case, and the packaged chip is assembled on a PCB with other hardware. An adversary may introduce malicious hardware components around the genuine design to induce malfunctions or increase leakages.
\paragraph*{}
\textbf{Post-silicon test:} During the testing phase, an adversary can no longer modify the genuine circuit structure. However, the test setup, programs, or reports may be altered to mask potential Trojan effects. Additionally, as the final step in the IC design process, this phase represents the last opportunity for original designers to detect Trojans before the deployment stage.
\subsection{ Abstraction Level}
\paragraph*{}
The abstraction level pertains to the potential manipulation of the design in the event that an adversary gains access to sensitive files across various abstraction levels. The ensuing examination highlights opportunities for HT insertion at each abstraction level.
\paragraph*{}
\textbf{System level}: A Hardware Trojan (HT) can involve modifications to functional specifications, protocols, interfaces, and constraints within the authentic design. If an adversary operates at the system level, they might introduce ambiguous specifications to gain control over confidential data transmitted by the manufactured device. For example, during the specification phase, an adversary could manipulate the specifications of true random number generators (TRNG), causing them to operate predictably under specific conditions known only to the HT owner. Such alterations have the potential to significantly compromise the reliability of secure systems relying on these architectures, enabling attackers to access sensitive information.
\paragraph*{}
\textbf{Development environment level:} Tools and scripts that are not trusted may contain concealed functionalities, causing designers to create circuits that are compromised by Trojans. Moreover, untrusted simulation tools and testbenches have the potential to obscure Hardware Trojan (HT) effects. Any untrustworthy third-party vendor could introduce Trojans at this particular level. 
\paragraph*{}
\textbf{Register-transfer level:} A Hardware Trojan (HT) can manifest as straightforward alterations in authentic Register-Transfer (RT) level codes or constraint files. An adversary may manipulate circuit functions to induce significant outcomes, such as failures in cryptographic blocks. Potential sources for HT insertion at this level include attackers during the design phase or an untrusted supplier of code.
\paragraph*{}
\textbf{Gate level:} The inclusion or removal of one or more gates within the initial netlist is classified as a gate-level Hardware Trojan (HT). The standard delay format (SDF) files, which encompass system timing data, can also be altered to modify timing checks, constraints, and delays, concealing the effects of the HT. Adversaries operating during the gate design phase and third-party vendors possess the capability to introduce Trojans at this specific level.
\paragraph*{}
\textbf{Transistor level:} The introduction of additional transistors can substantially raise leakages, providing attackers with insights into the internal states of security-focused circuits. Additionally, the incorporation of transistors may be employed to extend critical path delays, causing malfunctions in the circuit. Possible origins of Trojans at this level include adversaries during the design phase or the utilization of untrustworthy tools, libraries, and models.
\paragraph*{}
\textbf{Physical layout level:} The initial parameters of circuit components remain susceptible even following layout generation. An example is an attacker manipulating original masks, modifying transistor dimensions like lengths, widths, or channel doping concentrations. Additionally, resizing wires can cause malfunctions and increased leakages. Adversaries operating at both the design and fabrication levels, as well as third-party mask shops, have the capability to modify the original layout and introduce such Trojans.
\subsection{Some employed strategies in the insertion of HTs}
\paragraph*{}
In emulating the strategies employed by adversaries in the insertion of hardware Trojan horses (HTH), Alkabani and Koushanfar categorized the necessary components into three groups: trigger, storage, and driver. A trigger serves to activate the intended HTH, and following the trigger event, the ensuing action can be recorded in either memory or a sequential circuit. The driver is responsible for executing the action prompted by the trigger. Based on this classification, Alkabani and Koushanfar propose a systematic approach for embedding hardware Trojans into integrated circuits (ICs) through pre-synthesis manipulation of the circuit's structure \cite{4559059}. This model addresses trust concerns related to Intellectual Property (IP) cores, especially when multiple cores from various vendors are utilized. In an abstracted view of the design process, the Trojan designer formulates a high-level design description to ascertain the computation model of the circuit that can be represented by a finite-state machine (FSM).
\paragraph*{}
Tiago D. Perez and Samuel Pagliarini have presented a Hardware Trojan Insertion in Finalized Layouts \cite{9956883}. Their work provides the first-ever documentation of how effortlessly an HT can be incorporated into a finalized layout. This is achieved by introducing an insertion framework based on the engineering change order flow. To validate their findings, they have constructed an ASIC prototype using 65-nm CMOS technology, featuring four trojaned cryptocores. In each core, a side-channel HT is inserted with the goal of leaking the cryptokey over a power channel.
\paragraph*{}
Yu Liu et al. have presented a wireless cryptographic IC containing two hardware Trojans capable of leaking the encryption key \cite{7792718}. Using silicon measurements from 40 chips fabricated in Taiwan Semiconductor Manufacturing Company's (TSMC's) 0.35-$\mu$ m technology, they demonstrate the operation of two hardware Trojans. These Trojans are designed to leak the secret key of a wireless cryptographic integrated circuit (IC) that includes an Advanced Encryption Standard (AES) core and an ultrawideband (UWB) transmitter (TX). Their impact is carefully concealed within the transmission specification margins allowed for process variations. These hardware Trojans evade detection by production testing methods for both the digital and analog parts of the IC, and they do not violate the transmission protocol or any system-level specifications. However, an informed adversary who knows what to look for in the transmission power waveform can retrieve the 128-bit AES key, leaked with every 128-bit ciphertext block sent by the UWB TX.
\paragraph*{}
Exurville et al., have described the structure of created HTs and how they are inserted at layout level in FPGA \cite{7092492}. The attack scenario involves an untrusted ASIC foundry. When the tape-out database (GDS file) is received, the perpetrator introduces a Hardware Trojan (HT) before the fabrication process. To minimize the impact of the HT on the authentic circuit, it is crucial to insert the HT while preserving the original placement and routing of the target circuit. For simulating HT insertion on ASICs, maintaining identical placement and routing between the golden circuit and the HT-infected circuit on FPGAs is essential. Consequently, the only disparities between the two lie in the logic and interconnect used by the HT.
\paragraph*{}
To insert an HT without altering the routing, the following steps are executed within the Xilinx framework:
\begin{enumerate}
	\item Synthesize, translate, map, place and route the original circuit, which, in this case, is an AES-128 block cipher.
	\item Extract the Native Circuit Description (NCD) file containing all the circuit, placement, and routing details of the original circuit (the golden model).
	\item Utilize the FPGA Editor tool to open the NCD file and manually or through a script, insert the HT in unused LUTs and Slices of the FPGA.
	\item Generate bit files for both the original and infected circuits using FPGA Editor.
\end{enumerate}
\paragraph*{}
This method ensures that the placement and routing remain consistent in both the golden and HT-infected circuits, facilitating the proof-of-concept of the HT attack at the ASIC layout level, where the FPGA fabric is considered as an ASIC.
\paragraph*{}
Yang et al. have developed an Analog malicious hardware trojan \cite{7546493}. They demonstrate how a fabrication-time attacker can utilize analog circuits to create a hardware attack that is both small (requiring as little as one gate) and stealthy (needing an unlikely trigger sequence before affecting a chip's functionality). In the open spaces of an already placed and routed design, a circuit is constructed using capacitors to siphon charge from nearby wires during transitions between digital values. When the capacitors fully charge, they execute an attack that compels a victim flip-flop to assume a desired value. The attack is weaponized into a remotely-controllable privilege escalation by connecting the capacitor to a controllable wire and selecting a victim flip-flop that stores the privilege bit for their processor. This attack is implemented in an OR1200 processor, and a chip is fabricated as part of the process.
\paragraph*{}
Cruz et al. have presented A Machine Learning Based Automatic Hardware Trojan Attack Space Exploration and Benchmarking Framework \cite{cruz2022machine}. They introduce MIMIC, a pioneering machine learning-guided framework for automated Trojan insertion. This framework has the ability to generate a substantial and targeted set of valid Trojans for a given design by emulating the characteristics of a small group of known Trojans. While existing tools can automatically insert Trojan instances using fixed Trojan templates, they lack the capability to analyze known Trojan attacks to create new instances that precisely capture the threat model. MIMIC operates in two primary steps: (1) it examines the structural and functional features of existing Trojan populations in a multi-dimensional space to train machine learning models and generate numerous "virtual Trojans" for the specified design, (2) subsequently, it integrates them into the design by aligning their functional/structural properties with suitable nets of the internal logic structure. 
\section{Detection Techniques}
\paragraph*{}
Hardware Trojan detection is a critical aspect of ensuring the security and reliability of integrated circuits and electronic systems. The primary goal of hardware Trojan detection is to identify the presence of such malicious entities within a chip and prevent their adverse impact on system performance. Several detection techniques and methodologies have been developed to address the growing sophistication of Hardware Trojans.
\paragraph*{}
The methods essentially assess the deviations induced by Hardware Trojans (HTs) on the system's behavior or seek potential HT profiles. In order to achieve this, designers need to be familiar with at least one specific parameter from the authentic device or define a target HT model for detection. If the deviation observed in the assessed parameter of a design under Trojan test (DUTT) surpasses an acceptable margin, the DUTT is categorized as being infected by a Trojan. A scheme derived from earlier surveys and research outlines the primary classifications of testing methods employed for identifying Trojans.
\paragraph*{}
The majority of current methods focus on detecting Trojans in manufactured integrated circuits (ICs) and presume the presence of a gate-level golden netlist. There have been limited studies addressing Trojan detection at higher-level design descriptions, such as the register transfer level IP. In reference \cite{smith2007detecting}, a method based on structural checking is proposed for validating the integrity of third-party intellectual property (IP). However, this technique faces challenges in scalability when applied to extensive designs \cite{abramovici2009integrated}. It's important to highlight that there is currently no singular, universally effective technique that can be employed to detect all types of Trojans.
\paragraph*{}
Trojan detection methods can be categorized into two primary types: destructive and non-destructive. Destructive techniques, as outlined in references [36-37], involve employing a subset of manufactured integrated circuits (ICs) that undergo de-metallization using Chemical Mechanical Polishing (CMP)\footnote{Chemical mechanical polishing (CMP) is a planarization technique that was developed for semiconductor applications in the late 1980s and early 1990s. During this period, the number of metal layers increased dramatically and device topographies began to exhibit features that inhibited conformal deposition and gap fill by photoresist, metal, and insulator films. For more info visit: https://www.mks.com/n/chemical-mechanical-polishing, accessed 04 February 2024.}, followed by Scanning Electron Microscope (SEM) image reconstruction and analysis [3].
\section{Software and Hardware Materials in Trojan Mitigation}
\paragraph*{}